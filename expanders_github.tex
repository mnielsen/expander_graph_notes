\documentclass[12pt]{article}

\begin{document}

\title{Introduction to expander graphs} 

\author{Michael A. Nielsen\thanks{mn@michaelnielsen.org and
  http://michaelnielsen.org}}

\date{\today}

\maketitle

These notes introduce one of the deepest ideas in modern computer
science, \emph{expander graphs}.  Expanders are one of those powerful
ideas that crop up in many apparently unrelated contexts, and that
have a phenomenally wide range of uses.  The goal of the posts is to
explain what an expander is, and to learn just enough about them that
we can start to understand some of their uses.  The notes require a
little background in graph theory, computer science, linear algebra
and Markov chains (all at about the level of a first course) to be
comprehensible.  I am not an expert on expanders, and these notes are
just an introduction.  They are are mostly based on some very nice
2003 lecture notes by Nati Linial and Avi Wigderson, available on the
web at http://www.math.ias.edu/\~boaz/ExpanderCourse/.

The notes are based on a series of blog posts I wrote about expanders,
which are available online at
http://michaelnielsen.org/blog/expander-graphs-the-complete-notes/.
I'm now releasing these notes under a Creative Commons Attribution
license (CC BY 3.0).  That means anyone can copy, distribute, transmit
and adapt/remix the work, provided my contribution is attributed.

\section{Introduction to expanders}

\emph{Expander graphs} are one of the deepest tools of theoretical
computer science and discrete mathematics, popping up in all sorts of
contexts since their introduction in the 1970s.  Here's a list of some
of the things that expander graphs can be used to do.  Don't worry if
not all the items on the list make sense: the main thing to take away
is the sheer \emph{range} of areas in which expanders can be applied.

\begin{itemize}
  
\item \emph{Reduce the need for randomness:} That is, expanders can be
  used to reduce the number of random bits needed to make a
  probabilistic algorithm work with some desired probability.
  
\item \emph{Find good error-correcting codes:} Expanders can be used
  to construct error-correcting codes for protecting information
  against noise.  Most astonishingly for information theorists,
  expanders can be used to find error-correcting codes which are
  efficiently encodable and decodable, with a non-zero rate of
  transmission.  This is astonishing because finding codes with these
  properties was one of the holy grails of coding theory for
  \emph{decades} after Shannon's pioneering work on coding and
  information theory back in the 1940s.
  
\item \emph{A new proof of PCP:} One of the deepest results in
  computer science is the PCP theorem, which tells us that for all
  languages $L$ in \textbf{NP} there is a randomized polyonomial-time
  proof verifier which need only check a \emph{constant} number of
  bits in a purported proof that $x \in L$ or $x \not \in L$, in order
  to determine (with high probability of success) whether the proof is
  correct or not.  This result, originally established in the earlier
  1990s, has recently been given a new proof based on expanders.

\end{itemize}

What's remarkable is that none of the topics on this list appear to be
related, \emph{a priori}, to any of the other topics, nor do they
appear to be related to graph theory.  Expander graphs are one of
these powerful unifying tools, surprisingly common in science, that
can be used to gain insight into an an astonishing range of apparently
disparate phenomena.

I'm not an expert on expanders.  I'm writing these notes to help
myself (and hopefully others) to understand a little bit about
expanders and how they can be applied.  I'm not learning about
expanders with any specific intended application in mind, but rather
because they seem to behind some of the deepest insights we've had in
recent years into information and computation.

What is an expander graph?  Informally, it's a graph $G = (V,E)$ in
which every subset $S$ of vertices \emph{expands} quickly, in the
sense that it is connected to many vertices in the set $\overline S$
of complementary vertices.  Making this definition precise is the main
goal of the remainder of this section.

Suppose $G = (V,E)$ has $n$ vertices.  For a subset $S$ of $V$ we
define the \emph{edge boundary} of $S$, $\partial S$, to be the set of
edges connecting $S$ to its complement, $\overline S$.  That is,
$\partial S$ consists of all those edges $(v,w)$ such that $v \in S$
and $w \not \in S$.  The \emph{expansion parameter for $G$} is defined
by
\begin{eqnarray}
  h(G) \equiv \min_{S: |S| \leq n/2} \frac{|\partial S|}{|S|},
\end{eqnarray}
where $|X |$ denotes the size of a set $X$.

One standard condition to impose on expander graphs is that they be
$d$-regular graphs, for some constant $d$, i.e., they are graphs in
which every vertex has the same degree, $d$.  I must admit that I'm
not entirely sure why this $d$-regularity condition is imposed. One
possible reason is that doing this simplifies a remarkable result
which we'll discuss later, relating the expansion parameter $h(G)$ to
the \emph{eigenvalues} of the adjacency matrix of $G$.  (If you don't
know what the adjacency matrix is, we'll give a definition later.)

\textbf{Example:} Suppose $G$ is the complete graph on $n$ vertices,
i.e., the graph in which every vertex is connected to every other
vertex.  Then for any vertex in $S$, each vertex in $S$ is connected
to \emph{all} the vertices in $\overline S$, and thus $|\partial S| =
|S| \times |\overline S| = |S|(n-|S|)$.  It follows that the expansion
parameter is given by
\begin{eqnarray}
  h(G) = \min_{S: |S|\leq n/2} n-|S| = \left\lceil \frac{n}{2} \right\rceil.
\end{eqnarray}

For reasons I don't entirely understand, computer scientists are most
interested in the case when the degree, $d$, is a small constant, like
$d = 2,3$ or $4$, not $d=n-1$, as is the case for the complete graph.
Here's an example with constant degree.

\textbf{Example:} Suppose $G$ is an $n \times n$ square lattice in $2$
dimensions, with periodic boundary conditions (so as to make the graph
$4$-regular).  Then if we consider a large connected subset of the
vertices, $S$, it ought to be plausible that that the edge boundary
set $\partial S$ contains roughly one edge for each vertex on the
perimeter of the region $S$.  We expect there to be roughly
$\sqrt{|S|}$ such vertices, since we are in two dimensions, and so
$|\partial S|/|S| \approx 1/\sqrt{|S|}$.  Since the graph can contain
regions $S$ with up to $O(n^2)$ vertices, we expect
\begin{eqnarray}
  h(G) = O\left( \frac{1}{n} \right)
\end{eqnarray}
for this graph.  I do not know the exact result, but am confident that
this expression is correct, up to constant factors and higher-order
corrections.  It'd be a good exercise to figure out exactly what
$h(G)$ is.  Note that as the lattice size is increased, the expansion
parameter decreases, tending toward $0$ as $n\rightarrow \infty$.

\textbf{Example:} Consider a random $d$-regular graph, in which each
of $n$ vertices is connected to $d$ other vertices, chosen at random.
Let $S$ be a subset of at most $n/2$ vertices.  Then a typical vertex
in $S$ will be connected to roughly $d \times |\overline S|/n$
vertices in $\overline S$, and thus we expect $|\partial S| \approx d
\times |S| |\overline S|/n$, and so
\begin{eqnarray}
  \frac{|\partial S|}{|S|} \approx d \frac{|\overline S|}{n}.
\end{eqnarray}
Since $|\overline S|$ has its minimum at approximately $n/2$ it
follows that $h(G) \approx d/ 2$, independent of the size $n$.

\textbf{Exercise:} Show that a disconnected graph always has expansion
parameter $0$.

In each of our examples, we haven't constructed just a single graph,
but rather an entire family of graphs, indexed by some parameter $n$,
with the property that as $n$ gets larger, so too does the number of
vertices in the graph.  Having access to an entire family in this way
turns out to be much more useful than having just a single graph, a
fact which motivates the definition of expander graphs, which we now
give.

Suppose we have a family $G_j = (V_j,E_j)$ of $d$-regular graphs,
indexed by $j$, and such that $|V_j| = n_j$ for some increasing
function $n_j$.  Then we say that the family $\{ G_j \}$ is a
\emph{family of expander graphs} if the expansion parameter is bounded
strictly away from $0$, i.e., there is some small constant $c$ such
that $h(G_j) \geq c > 0$ for all $G_j$ in the family.  We'll often
abuse nomenclature slightly, and just refer to the expander $\{ G_j
\}$, or even just $G$, omitting explicit mention of the entire family
of graphs.

\section{Explicit examples of expanders}

We've seen previously that a family of $d$-regular random graphs on
$n$ vertices defines an expander.  For applications it is often more
useful to have more explicit constructions for expanders.  In
particular, for applications to algorithms it is often useful to
construct expanders on $O(2^n)$ vertices, where $n$ is some parameter
describing problem size.  Just to store a description of a random
graph on so many vertices requires exponentially much time and space,
and so is not feasible.  Fortunately, more parsimonious constructions
are possible, which we now describe.

\textbf{Example:} In this example the family of graphs is indexed by a
prime number, $p$.  The set of vertices for the graph $G_p$ is just
the set of points in $Z_p$, the field of integers modulo $p$.  We
construct a $3$-regular graph by connecting each vertex $x \neq 0$ to
$x-1,x+1$ and $x^{-1}$.  The vertex $x=0$ is connected to $p-1,0$ and
$1$.  According to the lecture notes by Linial and Wigderson, this was
proved to be a family of expanders by Lubotsky, Phillips and Sarnak in
1988, but I don't know a lower bound on the expansion parameter.  Note
that even for $p = O(2^n)$ we can do basic operations with this graph
(e.g., random walking along its vertices), using computational
resources that are only polynomial in time and space.  This makes this
graph potentially far more useful in applications than the random
graphs considered earlier.

\textbf{Example:} A similar but slightly more complex example is as
follows.  The vertex set is $Z_m \times Z_m$, where $m$ is some
positive integer, and $Z_m$ is the additive group of integers modulo
$m$.  The degree is $4$, and the vertex $(x,y)$ has edges to $(x\pm
y,y)$, and $(x,x \pm y)$, where all addition is done modulo $m$.
Something which concerns me a little about this definition, but which
I haven't resolved, is what happens when $m$ is even and we choose $y
= m/2$ so that, e.g., the vertices $(x+y,y)$ and $(x-y,y)$ coincide
with one another.  We would expect this duplication to have some
effect on the expansion parameter, but I haven't thought through
exactly what.

%#post_footer: Unfortunately, I don't know any more examples of
%# expanders, although I'm sure there are some!  Still, hopefully you 
%# will agree that these examples are
%# pretty nice: they are easy to describe, and allow us to work
%# with expander graphs on exponentially large vertex sets, without
%# expending too much effort.  I haven't proved that these families are
%# expanders: doing that requires some extra technical tools based
%# on the <em>adjacency matrix</em> of a graph, which is the subject of
%# the next post.
%#end_post_footer

%#post_break

%#post_header: <strong>Today's post</strong> digresses to explain one of
%# the most interesting ideas in graph theory, namely, to describe a graph
%# through its <em>adjacency matrix</em>.   By describing a graph in
%# terms of matrices we enable all the powerful tools of linear algebra
%# to be brought to bear on graph theory, and we'll see that there are all
%# sorts of interesting connections between graphs and linear algebraic
%# concepts such as eigenvalues.  The connection to expanders will be
%# explained in later posts, where we'll see that the expansion parameter
%# is connected to the second largest eigenvalue of the adjacency matrix.
%# I believe (but have not checked) that it is this result which is
%# used to establish that the graphs described in the previous post
%# are actually expanders.
%#end_post_header

\section{Graphs and their adjacency matrices}

How can we prove that a family of graphs is an expander?  Stated
another way, how does the expansion parameter $h(G)$ vary as the graph
$G$ is varied over all graphs in the family?

One way of tackling the problem of computing $h(G)$ is to do a brute
force calculation of the ratio $|\partial S|/|S|$ for every subset $S$
of vertices containing no more than half the vertices in the graph.
Doing this is a time-consuming task, since if there are $n$ vertices
in the graph, then there are exponentially many such subsets $S$.

\textbf{Problem:} In general, how hard is it to find the subset $S$
minimizing $|\partial S|/|S|$?  Can we construct an
\textbf{NP-Complete} variant of this problem?  I don't know the answer
to this question, and I don't know if anyone else does, either.

Fortunately, there is an extraordinarily beautiful approach to the
problem of determining $h(G)$ which is far less computationally
intensive.  It involves the \emph{adjacency matrix} $A(G)$ of the
graph $G$.  By definition, the rows and columns of the adjacency
matrix are labelled by the vertices of $V$.  For vertices $v$ and $w$
the entry $A(G)_{vw}$ is defined to be $1$ if $(v,w)$ is an edge, and
$0$ if it is not an edge.

It is a marvellous fact that properties of the \emph{eigenvalue
  spectrum} of the adjacency matrix $A(G)$ can be used to understand
properties of the graph $G$.  This occurs so frequently that we refer
to the spectrum of $A(G)$ as \emph{the spectrum of the graph $G$}.  It
is useful because the eigenvalue spectrum can be computed quickly, and
certain properties, such as the largest and smallest eigenvalue, the
determinant and trace, can be computed extremely quickly.

More generally, by recasting graphs in terms of adjacency matrices, we
open up the possibility of using tools from linear algebra to study
the properties of graphs.  Although we're most interested in studying
expanders, for the rest of this section I'm going to digress from the
study of expanders, studying how the linear algebraic point of view
can help us understand graphs, without worrying about how this
connects to expanders. This digression is partially motivated by the
fact that this is beautiful stuff (at least in my opinion), and is
partially because our later discussion of expanders will be based on
this linear algebraic point of view, and so it's good to get
comfortable with this point of view.

The following exercise provides a good example of how graph properties
can be related to the eigenvalues of the graph.

\textbf{Exercise:} Prove that if two graphs are isomorphic, then they
have the same spectrum.

This result is often useful in proving that two graphs are not
isomorphic: simply compute their eigenvalues, and show that they are
different.  A useful extension of the exercise is to find an example
of two graphs which have the same spectra, but which are not
isomorphic.

Note that the adjacency matrix may be considered as a matrix over any
field, and the result of the exercise is true over any field.  (I've
often wondered if the converse is true, but don't know the answer.)
Nonetheless, by and large, we'll consider the adjacency matrix as a
matrix over the field $R$ of real numbers.  Assuming that $G$ is an
undirected graph, we see that $A(G)$ is a real symmetric matrix, and
thus can be diagonalized.  We will find it convenient to write the
eigenvalues of a graph $G$ in non-increasing order, as $\lambda_1(G)
\geq \lambda_2(G) \geq \ldots \geq \lambda_n(G)$.  

A fact we'll make a lot of use of is that when $G$ is $d$-regular the
largest eigenvalue of $G$ is just $d$.  To see this, note that the
vector $\vec 1 \equiv (1,1,\ldots,1)$ is an eigenvector of $G$ with
eigenvalue $d$.  To prove that $d$ is the largest eigenvalue seems to
be a little bit harder.  We'll just sketch a proof.  To prove this it
is sufficient to show that $v^T A(G) v \leq d$ for all normalized
vectors $v$.  From the $d$-regularity of $G$ it follows that $A(G)/d$
is a doubly stochastic matrix, i.e., has non-negative entries, and all
rows and columns sum to one.  A theorem of Birkhoff ensures that
$A(G)/d$ can be written as a convex combination of permutation
matrices, so $A(G) = d \sum_j p_j P_j$, where $p_j$ are probabilities,
and the $P_j$ are permutation matrices.  This gives $v^T A(G) v = d
\sum_j p_j v^T P_j v$.  But $v^T P_j v \leq 1$ for any permutation
$P_j$, which gives the desired result.

The following proposition gives another example of the relationships
one can find between a graph and its spectrum.

\textbf{Proposition:} A $d$-regular graph $G$ is connected if and only
if $\lambda_1(G) > \lambda_2(G)$.

\textbf{Proof:} The easy direction is the reverse implication, for
which we prove the contrapositive, namely, that a $d$-regular
disconnected graph has $\lambda_1(G) = \lambda_2(G)$.  This follows by
breaking $G$ up into disconnected components $G_1$ and $G_2$, and
observing that $A(G) = A(G_1) \oplus A(G_2)$, where $\oplus$ is the
matrix direct sum.  Since both $G_1$ and $G_2$ are $d$-regular it
follows that they both have maximal eigenvalue $d$, and so $d$ appears
at least twice in the spectrum of $A(G)$.

At the moment, I don't see an easy way of proving the forward
implication.  One not very satisfying proof is to observe that
$A(G)/d$ is the Markov transition matrix for a random walk on the
graph, and that since the graph is connected, the random walk must
converge to a unique distribution, which implies that in the limit of
large $n$ there can only be one vector $v$ such that $(G^n/d^n) v =
v$.  This means that $G^n$'s largest eigenvalue is non-degenerate,
from which it follows that $G$'s largest eigenvalue is non-degenerate.
This is a sketch, but it can all be established rigorously with a
little work and the aid of well-known theorems on Markov chains.

The proof sketched in the previous paragraph is not really
satisfactory, since it involves an appeal to theorems which are in
some sense less elementary than the result under discussion.  Another
possibility which I've explored but haven't made work with complete
rigour is to investigate $G^n/d^n$ more explicitly.  With a little
thought one can prove that the entry $G^n_{vw}$ is just the number of
paths between $v$ and $w$ of length $n$.  Since $G$ is connected, we'd
expect in the limit of large $n$ this number would be dominated by a
term which does not depend on $w$, and would just scale like the total
number of paths of length $n$ starting at $v$ (which is $d^n$),
divided by the total number of possible destinations $w$, which is
$n$, giving $G^n_{vw}/d^n \rightarrow 1/n$.  (This would only be true
if $G$ has self-loops $(v,v)$.)  Of course, the matrix whose entries
are all $1/n$ has a single eigenvalue $1$, with all the rest $0$,
which would suffice to establish the theorem.

\textbf{QED}

\textbf{Problem:} How should we interpret the determinant of a graph?
What about the trace?

\textbf{Problem:} If we consider $A(G)$ as a matrix over the field
$Z_2 = \{0,1\}$, then it is possible to define a matrix sum $G_1+G_2$,
whose adjacency matrix is just $A(G_1)+A(G_2)$, and a matrix product
$G_1 \times G_2$ whose adjacency matrix is just $A(G_1) A(G_2)$.  Many
questions naturally suggest themseves: (1) when is there an edge
between $v$ and $w$ in $G_1+G_2$; (2) when is there an edge between
$v$ and $w$ in $G_1 \times G_2$ (these first two questions are easy to
answer); (3) for which graphs is $A(G)$ invertible, and thus a natural
inverse graph $G^{-1}$ exists; (4) how can we interpret the inverse
graph; (5) when do two graphs commute?

\textbf{Problem:} Along similar lines to the previous problem, it's
possible to define a tensor product of graphs.  What are the
properties of the graph tensor product?

The ideas I've described in this section are examples of the important
general principle that once you've defined a mathematical object, you
should seek out alternate representations (or even just partial
representations) of that object in terms of mathematical objects that
you already understand.  By recasting graphs as matrices, we open up
the possibility of using all the tools of linear algebra to answer
questions about graphs.  This can work in one of two ways: we can ask
a question about graphs, and try to see if it's possible to give a
linear algebraic answer, or we can ask what implication known results
of linear algebra have for graphs --- what does the Gaussian
elimination procedure correspond to, or the spectral decomposition, or
two matrices commuting, or the wedge product, or whatever.  Exploring
such connections has the potential to greatly enrich both subjects.

%#post_footer: In this post we've started to get a feel for how the 
%# properties of
%# graphs can be studied using linear algebra.  In the next post we'll
%# turn our attention back to expanders, and understand how the expansion
%# coefficient can be related to the gap [tex]\lambda_1(G)-\lambda_2(G)[/tex]
%# between the largest and second largest eigenvalues of $G$. 
%#end_post_footer

%#post_break

%#post_header: <strong>In today's post</strong> we really reap the benefits of
%# recasting graph theory in linear algebraic terms, showing that the
%# the expansion parameter can be related to the <em>gap</em> between
%# the largest and second largest eigenvalue of the graph.  This ought
%# to surprise you, perhaps even shock you.  It's a highly
%# non-trivial result, with the big payoff of relating something we
%# didn't really understand all that well (the expansion parameter) to
%# eigenvalue gaps, about which an enormous amount it known.
%#end_post_header

\section{Expansion and the eigenvalue gap}

Let's return our attention to expander graphs, and see what the
eigenvalues of a graph have to do with its expansion parameter.  We
define the \emph{gap for the graph $G$} to be the difference
$\Delta(G) \equiv \lambda_1(G)-\lambda_2(G)$ between the largest and
second-largest eigenvalues.  The expansion parameter and the gap are
connected by the following theorem:

\textbf{Theorem:} The expansion parameter $h(G)$ for a $d$-regular
graph $G$ is related to the gap $\Delta(G)$ by:
\begin{eqnarray}
  \frac{\Delta(G)}{2} \leq h(G) \leq \sqrt{2d \Delta(G)}.
\end{eqnarray}

Thus, properties of the eigenvalue gap can be used to deduce
properties of the expansion parameter.  For example, if the eigenvalue
gap for a family of $d$-regular graphs is bounded below by a positive
constant, then the expansion parameter must also be bounded below by a
positive constant, and so the family is an expander.

One reason for finding the connection between the gap and the
expansion parameter interesting is that it is far easier to estimate
the gap of an $n$ by $n$ matrix than it is to enumerate the
exponentially many subsets $S$ of the vertex set $V$, and compute
$|\partial S|/|S|$ for each one.

\textbf{Proof discussion:} We already understand that $\lambda_1(G) =
d$ for this graph, with eigenvector $\vec 1 = (1,1,\ldots,1)$.  So
we'll concentrate on trying to understand the behaviour of the second
largest eigenvalue, $\lambda_2(G)$.  The theorem tells us that the
difference between $d$ and $\lambda_2(G)$ is controlled both above and
below by the expansion parameter $h(G)$.

How can we get control over the second largest eigenvalue of $G$?  One
way is to observe that $\lambda_2(G)$ is just the maximum of the
expression $v^T A v / v^T v$, where $A$ is the adjacency matrix of
$G$, and we maximize over all vectors $v$ orthogonal to the
eigenvector $\vec 1$.  An encouraging fact is that this expression is
quite easy to deal with, because the condition that $v$ be orthogonal
to $\vec 1$ is actually equivalent to the sum of $v$'s entries being
equal to $0$, so we have
\begin{eqnarray}
  \lambda_2(G) = \max_{v: {\rm tr}(v) = 0} \frac{v^T A v}{v^T v},
\end{eqnarray}
where ${\rm tr}(v)$ is just the sum of the entries of the vector $v$.

We're going to provide a \emph{lower bound} on $\lambda_2(G)$ by
simply guessing a good choice of $v$ satisfying $\mbox{tr}(v) = 0$,
and using the fact that
\begin{eqnarray}
  \lambda_2(G) \geq \frac{v^T A v}{v^T v}.
\end{eqnarray}
To make a good guess, it helps to have a way of thinking about
expressions like $v^T A v$, where $\mbox{tr}(v) = 0$.  A convenient
way of thinking is to rewrite $v$ as the difference of two disjoint
probability distributions, $p$ and $q$, i.e., $v = p-q$, where $p$ and
$q$ are non-negative vectors each summing to $1$, and with disjoint
support.  This results in terms like $p^T A q$, which we can think of
in terms of transition probabilities between $q$ and $p$.  This will
allow us to apply the expansion properties of the graph.

Let's make these ideas a little more concrete.  The key is to define
$\vec 1_S$ to be the vector whose entries are $1$ on $S$, and $0$
elsewhere, and to observe that
\begin{eqnarray}
  \vec 1_S^T A \vec 1_T = |E(S,T)|,
\end{eqnarray}
where $|E(S,T)|$ is the number of edges between the vertex sets $S$
and $T$.  This suggests that we should choose $p$ and $q$ in terms of
vectors like $\vec 1_S$, since it will enable us to relate expressions
like $v^T A v$ to the sizes of various edge sets, which, in turn, can
be related to the expansion parameter.

Suppose in particular that we choose
\begin{eqnarray}
  v = \frac{\vec 1_S}{|S|}-\frac{\vec 1_{\overline S}}{|\overline S|}.
\end{eqnarray}
This satisfies the condition $\mbox{tr}(v) = 0$, and gives
\begin{eqnarray}
  v^T v = \frac{1}{|S|}+\frac{1}{|\overline S|}
\end{eqnarray}
and
\begin{eqnarray}
  v^T A v = \frac{1}{|S|^2} E(S,S) + \frac{1}{|\overline S|^2}
E(\overline S,\overline S) - \frac{2}{|S||\overline S|} E(S,\overline S).
\end{eqnarray}
The definition of an expander graph gives us control over
$E(S,\overline S)$, so it is convenient to rewrite $E(S,S)$ and
$E(\overline S,\overline S)$ in terms of $E(S,\overline S)$, using the
$d$-regularity of the graph:
\begin{eqnarray}
  E(S,S)+E(S,\overline S) = d |S|; \,\,\, E(\overline S,\overline S)+
E(S,\overline S) = d |\overline S|.
\end{eqnarray}
A straightforward substitution and a little algebra gives:
\begin{eqnarray}
  v^T A v = d \left( \frac{1}{|S|}+\frac{1}{|\overline S|} \right)
  - \left(\frac{1}{|S|}+\frac{1}{|\overline S|}\right)^2 E(S,\overline S).
\end{eqnarray}
Comparing with the earlier expression for the denominator $v^T v$, we
obtain
\begin{eqnarray}
  \lambda_2(G) \geq d-\left(\frac{1}{|S|}+\frac{1}{|\overline S|}\right)
  E(S,\overline S).
\end{eqnarray}
Now choose $S$ so that $E(S,\overline S) = h(G) |S|$, and $|S| \leq
n/2$, giving after a little algebra
\begin{eqnarray}
  \lambda_2(G) \geq d-2 h(G),
\end{eqnarray}
and thus
\begin{eqnarray}
  \frac{\Delta(G)}{2} \leq h(G),
\end{eqnarray}
which was the first of the two desired inequalities in the theorem.

The proof of the second inequality is a little more complicated.
Unfortunately, I haven't managed to boil the proof down to a form that
I'm really happy with, and for this reason I won't describe the
details.  If you're interested, you should try to prove it yourself,
or refer to the notes of Linial and Wigderson.

\textbf{QED}

\textbf{Problem:} Can we generalize this result so that it applies to
a general undirected graph $G$, not just to $d$-regular graphs?  Can
we prove an analogous statement for directed graphs, perhaps in terms
of singular values?  Can we define a a generalized notion of
"expansion" which can be applied to \emph{any} symmetric matrix $A$
with non-negative entries, and connect that notion of expansion to the
eigenvalue gap?  Can we generalize even further?  What happens if we
change the field over which the matrix is considered?

%#post_break

%#post_header: <strong>In today's post</strong> we'll look at a simple
%# application of expanders, showing that a random walk on an expander
%# graph is likely to quickly escape from any sufficiently small
%# subset of vertices.  Intuitively, of course, this result is not
%# surprising, but the exact quantitative form of the result turns out
%# to be extremely useful in the next post, which is about decreasing the number
%# of random bits used by a randomized algorithm.
%#end_post_header

\section{Random walks on expanders}

Many applications of expanders involve doing a random walk on the
expander.  We start at some chosen vertex, and then repeatedly move to
any one of the $d$ neighbours, each time choosing a neighbour
uniformly at random, and independently of prior choices.

To describe this random walk, suppose at some given time we have a
probability distribution $p$ describing the probability of being at
any given vertex in the graph $G$.  We then apply one step of the
random walk procedure described above, i.e., selecting a neighbour of
the current vertex uniformly at random.  The updated probability
distribution is easily verified to be:
\begin{eqnarray}
  p' = \frac{A(G)}{d} p.
\end{eqnarray}
That is, the Markov transition matrix describing this random walk is
just $\hat A(G) \equiv A(G)/d$, i.e., up to a constant of
proportionality the transition matrix is just the adjacency matrix.
This relationship between the adjacency matrix and random walks opens
up a whole new world of connections between graphs and Markov chains.

One of the most important connections is between the eigenvalues of
Markov transition matrices and the rate at which the Markov chain
converges to its stationary distribution.  In particular, the
following beautiful theorem tells us that when the uniform
distribution is a stationary distribution for the chain, then the
Markov chain converges to the uniform distribution exponentially
quickly, at a rate determined by the second largest eigenvalue of $M$.

\textbf{Exercise:} Show that if $M$ is a normal transition matrix for
a Markov chain then $1 = \lambda_1(M) \geq \lambda_2(M) \geq \ldots$.

\textbf{Theorem:} Suppose $M$ is a normal transition matrix for a
Markov chain on $n$ states, with the uniform distribution $u = \vec
1/n$ as a stationary point, $M u = u$.  Then for any starting
distribution $p$,
\begin{eqnarray}
  \| M^t p - u \|_1 \leq \sqrt{n} \lambda_2(M)^t,
\end{eqnarray}
where $\| \cdot \|_1$ denotes the $l_1$ norm.

The normality condition in this theorem may appear a little
surprising.  The reason it's there is to ensure that $M$ can be
diagonalized.  The theorem can be made to work for general $M$, with
the second largest eigenvalue replaced by the second largest singular
value.  However, in our situation $M$ is symmetric, and thus
automatically normal, and we prefer the statement in terms of
eigenvalues, since it allows us to make a connection to the expansion
parameter of a graph.  In particular, when $M = \hat A(G)$ we obtain:
\begin{eqnarray}
  \| \hat A(G)^t p-u\|_1 \leq \sqrt{n} \left(\frac{\lambda_2(G)}{d}\right)^t.
\end{eqnarray}
Combining this with our earlier results connecting the gap to the
expansion parameter, we deduce that
\begin{eqnarray}
  \| \hat A(G)^t p-u\|_1 \leq \sqrt{n} \left(1-\frac{h(G)^2}{2d^2}\right)^t.
\end{eqnarray}
Thus, for a family of expander graphs, the rate of convergence of the
Markov chain is exponentially fast in the number of time steps $t$.

\textbf{Exercise:} Suppose $M$ is a transition matrix for a Markov
chain.  Show that the uniform distribution $u$ is a stationary point
point for the chain, i.e., $Mu = u$, if and only if $M$ is doubly
stochastic, i.e., has non-zero entries, and all rows and columns of
the matrix sum to $1$.

\textbf{Proof:} We start by working with the $l_2$ norm $\| \cdot
\|_2$.  Since $Mu = u$ we have $M^t u = u$, and so:
\begin{eqnarray}
  \|M^t p - u \|_2 = \|M^t(p-u) \|_2.
\end{eqnarray}
A computation shows that $p-u$ is orthogonal to $u$.  But $u$ is an
eigenvector of $M$ with the maximum eigenvalue, $1$, and thus $p-u$
must lie in the span of the eigenspaces with eigenvalues
$\lambda_2(M),\lambda_3(M),\ldots$.  It follows that
\begin{eqnarray}
  \|M^t(p-u)\|_2 \leq \lambda_2(M)^t \|p-u\|_2 \leq \lambda_2(M)^t,
\end{eqnarray}
where we used the fact that $\| p-u\|_2 \leq 1$, easily established by
observing that $\|p-u\|_2$ is convex in $p$, and thus must be
maximized at an extreme point in the space of probability
distributions; the symmetry of $u$ ensures that without loss of
generality we may take $p = (1,0,\ldots,0)$.  To convert this into a
result about the $l_1$ norm, we use the fact that in $n$ dimensions
$\|v\|_1 \leq \sqrt{n} \|v\|_2$, and thus we obtain
\begin{eqnarray}
  \|M^t(p-u)\|_1 \leq \sqrt{n} \lambda_2(M)^t,
\end{eqnarray}
which was the desired result.  \textbf{QED}

What other properties do random walks on expanders have?  We now prove
another beautiful theorem which tells us that they ``move around
quickly'', in the sense that they are exponentially unlikely to stay
for long within a given subset of vertices, $B$, unless $B$ is very
large.

More precisely, suppose $B$ is a subset of vertices, and we choose
some vertex $X_0$ uniformly at random from the graph.  Suppose we use
$X_0$ as the starting point for a random walk, $X_0,\ldots,X_t$, where
$X_t$ is the vertex after the $t$th step.  Let $B(t)$ be the event
that $X_j \in B$ for \emph{all} $j$ in the range $0,\ldots,t$.  Then
we will prove that:
\begin{eqnarray}
  \mbox{Pr}(B(t)) \leq \left( \frac{\lambda_2(G)}{d} + \frac{|B|}{n}
    \right)^t 
\end{eqnarray}
Provided $\lambda_2(G)/d + |B|/n < 1$, we get the desired exponential
decrease in probability.  For a family of expander graphs it follows
that there is some constant $\epsilon > 0$ such that we get an
exponential decrease for any $B$ such that $|B|/n < \epsilon$.  These
results are special cases of the following more general theorem about
Markov chains.

\textbf{Theorem:} Let $X_0$ be uniformly distributed on $n$ states,
and let $X_0,\ldots,X_t$ be a time-homogeneous Markov chain with
transition matrix $M$.  Suppose the uniform distribution $u$ is a
stationary point of $M$, i.e., $Mu = u$.  Let $B$ be a subset of the
states, and let $B(t)$ be the event that $X_j \in B$ for all $j \in
0,\ldots,t$.  Then
\begin{eqnarray}
  \mbox{Pr}(B(t)) \leq \left( \lambda_2(M) + \frac{|B|}{n}
    \right)^t.
\end{eqnarray}

\textbf{Proof:} The first step in the proof is to observe that:
\begin{eqnarray}
  \mbox{Pr}(B(t)) = \|(PMP)^t P u \|_1,
\end{eqnarray}
where the operator $P$ projects onto the vector space spanned by those
basis vectors corresponding to elements of $B$.  This equation is not
entirely obvious, and proving it is a good exercise for the reader.

The next step is to prove that $\| PMP \| \leq \lambda_2(M)+|B|/n$,
where the norm here is the operator norm.  We will do this below, but
note first that once this is done, the result follows, for we have
\begin{eqnarray}
  \mbox{Pr}(B(t)) = \| (PMP)^t P u \|_1 \leq \sqrt{n} \| (PMP)^t P u \|_2
\end{eqnarray}
by the standard inequality relating $l_1$ and $l_2$ norms, and thus
\begin{eqnarray}
  \mbox{Pr}(B(t)) \leq \sqrt{n} \| PMP \|^t \| P u \|_2,
\end{eqnarray}
by definition of the operator norm, and finally
\begin{eqnarray}
  \mbox{Pr}(B(t)) \leq \left( \lambda_2(M)+\frac{|B|}{n} \right)^t,
\end{eqnarray}
where we used the assumed inequality for the operator norm, and the
observation that $\| P u \|_2 = \sqrt{|B|}/n \leq 1/\sqrt{n}$.

To prove the desired operator norm inequality, $\| PMP \| \leq
\lambda_2(M)+|B|/n$, suppose $v$ is a normalized state such that $\|
PMP \| = |v^T PMP v|$.  Decompose $Pv = \alpha u + \beta u_\perp$,
where $u_\perp$ is a normalized state orthogonal to $u$.  Since $\|P v
\|_2 \leq \|v \|_2 = 1$ we must have $|\beta| \leq 1$.  Furthermore,
multiplying $Pv = \alpha u + \beta u_\perp$ on the left by $nu^T$
shows that $\alpha = n u^T P v$.  It follows that $|\alpha|$ is
maximized by choosing $v$ to be uniformly distributed over $B$, from
which it follows that $|\alpha| \leq \sqrt{|B|}$.  A little algebra
shows that
\begin{eqnarray}
  v^T PMP v =  \alpha^2 u^T M u + \beta^2 u_\perp^T M u_\perp.
\end{eqnarray}
Applying $|\alpha| \leq \sqrt{|B|}$, $u^T M u = u^Tu = 1/n$, $|\beta|
\leq 1$, and $u_\perp^T M u_\perp \leq \lambda_2(M)$ gives
\begin{eqnarray}
  v^T P M P v \leq \frac{|B|}{n} + \lambda_2(M),
\end{eqnarray}
which completes the proof. \textbf{QED}

%#post_footer: The results in today's post are elegant, but qualitatively
%# unsurprising.  (Of course, having elegant quantitative statements of
%# results that are qualitatively clear is worthwhile in its own right!)
%# In the next post we'll use these ideas to develop a genuinely 
%# surprising application of
%# expanders, to reducing the number of random bits required by a
%# probabilistic algorithm in order to achieve a desired success probability.
%#end_post_footer

%#post_break

%#post_header: <strong>In today's post</strong> we explain how expander
%# graphs can be used to reduce the number of random bits needed by 
%# a randomized algorithm in order to achieve a desired success probability.
%# This post is the culmination of the series: we make use
%# of the fact, proved in the last post, that random walks on an 
%# expander are exponentially unlikely
%# to remain localized in any sufficiently large subset of vertices, a fact that
%# relies in turn on the connection, developed in earlier posts, between
%# the eigenavlue gap and the expansion parameter.
%#end_post_header


\section{Reducing the number of random bits required by an algorithm}

One surprising application of expanders is that they can be used to
reduce the number of random bits needed by a randomized algorithm in
order to achieve a desired success probability.

Suppose, for example, that we are trying to compute a function $f(x)$
that can take the values $f(x) = 0$ or $f(x) = 1$.  Suppose we have a
randomized algorithm $A(x,Y)$ which takes as input $x$ and an $m$-bit
uniformly distributed random variable $Y$, and outputs either $0$ or
$1$.  We assume that:
\begin{itemize}
\item $f(x) = 0$ implies $A(x,Y) = 0$ with certainty.
  
\item $f(x) = 1$ implies $A(x,Y) = 1$ with probability at least
  $1-p_f$.
\end{itemize}
That is, $p_f$ is the maximal probability that the algorithm fails, in
the case when $f(x) = 1$, but $A(x,Y) = 0$ is output by the algorithm.

An algorithm of this type is called a \emph{one-sided} randomized
algorithm, since it can only fail when $f(x) = 1$, not when $f(x) =
0$.  I won't give any concrete examples of one-sided randomized
algorithms here, but the reader unfamiliar with them should rest
assured that they are useful and important --- see, e.g., the book of
Motwani and Raghavan (Cambridge University Press, 1995) for examples.

As an aside, the discussion of one-sided algorithms in this section
can be extended to the case of randomized algorithms which can fail
when either $f(x) = 0$ or $f(x) = 1$.  The details are a little more
complicated, but the basic ideas are the same.  This is described in
Linial and Wigderson's lecture notes.  Alternately, extending the
discussion to this case is a good problem.

How can we descrease the probability of failure for a one-sided
randomized algoerithm?  One obvious way of decreasing the failure
probability is to run the algorithm $k$ times, computing
$A(x,Y_0),A(x,Y_1),\ldots,A(x,Y_{k-1})$.  If we get $A(x,Y_j) = 0$ for
all $j$ then we output $0$, while if $A(x,Y_j) = 1$ for at least one
value of $J$, then we output $f(x) = 1$.  This algorithm makes use of
$km$ bits, and reduces the failure probability to at most $p_f^k$.

Expanders can be used to substantially decrease the number of random
bits required to achieve such a reduction in the failure probability.
We define a new algorithm $A'$ as follows.  It requires a $d$-regular
expander graph $G$ whose vertex set $V$ contains $2^m$ vertices, each
of which can represent a possible $m$-bit input $y$ to $A(x,y)$.  The
modified algorithm $A'$ works as follows:
\begin{itemize}
\item Input $x$.

\item Sample uniformly at random from $V$ to generate $Y_0$.
  
\item Now do a $k-1$ step random walk on the expander, generating
  random variables $Y_1,\ldots, Y_{k-1}$.
  
\item Compute $A(x,Y_0),\ldots,A(x,Y_{k-1})$.  If any of these are
  $1$, output $1$, otherwise output $0$.
\end{itemize}
We see that the basic idea of the algorithm is similar to the earlier
proposal for running $A(x,Y)$ repeatedly, but the sequence of
independent and uniformly distributed samples $Y_0,\ldots,Y_{k-1}$ is
replaced by a random walk on the expander.  The advantage of doing
this is that only $m+k \log(d)$ random bits are required --- $m$ to
sample from the initial uniform distribution, and then $\log(d)$ for
each step in the random walk.  When $d$ is a small constant this is
far fewer than the $km$ bits used when we simply repeatedly run the
algorithm $A(x,Y_j)$ with uniform and independently generated random
bits $Y_j$.

With what probability does this algorithm fail? Define $B_x$ to be the
set of values of $y$ such that $A(x,y) = 0$, yet $f(x) = 1$.  This is
the ``bad'' set, which we hope our algorithm will avoid.  The
algorithm will fail only if the steps in the random walk
$Y_0,Y_1,\ldots,Y_{k-1}$ all fall within $B_x$.  From our earlier
theorem we see that this occurs with probability at most:
\begin{eqnarray}
  \left( \frac{|B_x|}{2^m} + \frac{\lambda_2(G)}{d} \right)^{k-1}.
\end{eqnarray}
But we know that $|B_x|/2^m \leq p_f$, and so the failure probability
is at most
\begin{eqnarray}
  \left( p_f + \frac{\lambda_2(G)}{d} \right)^{k-1}.
\end{eqnarray}
Thus, provided $p_f+\lambda_2(G)/d < 1$, we again get an exponential
decrease in the failure probability as the number of repetitions $k$
is increased.

\section{Conclusion}

These notes have given an introduction to expanders, and there's much
we haven't covered.  More detail and more applications can be found in
the online notes of Linial and Wigderson, or in the research
literature.  Still, I hope that these notes have given some idea of
why these families of graphs are useful, and of some of the powerful
connections between graph theory, linear algebra, and random walks.


\end{document}
